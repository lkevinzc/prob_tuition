\documentclass[a4paper]{article}

\usepackage{parskip} % Package to tweak paragraph skipping
\usepackage{tikz} % Package for drawing
\usepackage{amsmath}
\usepackage{amssymb}
\usepackage{hyperref}

\usepackage[a4paper,margin=1in]{geometry}
\usepackage{fancyhdr}
\pagestyle{fancy}


\title{EE2012/ST2334 Discussion 1}
\author{Liu Zichen}
\date{14 Feb, 2019}
\begin{document}

\maketitle

\begin{enumerate}
\item
\textbf{[sample space]}
The color of a single pixel on the screen can usually be controlled by R, G, B components. Suppose the intensity value of R, G, B can only be taken from $ \{0, 1, 2\}$.
\begin{enumerate}
    \item What's the sample space of the color of a single pixel?
    \item If a low-resolution image consists of $28 \times 28$ pixels, what's the size of sample space for such an image?
\end{enumerate}

\item
\textbf{[event probability]}
Think of an event that has probability $p(x)=\frac{\pi}{4}$.

\item
\textbf{[Law]}
Write down the De Morgan’s Law for 3 event case.

\item
\textbf{[counting]}
How the multiplication principle and addition principle can be represented using a tree diagram? Come up with an example to show a $(2 \times 6 \times 2)$ and a $(2 + 3)$ case.

\item
\textbf{[permutation]}
$P(n, k) = \frac { n ! } { ( n - k ) ! }$. Explain the equation on the left. Does the order matter in permutation?\\
When do we use $(n - 1)!$ to calculate permutation?\\
When do we use $P(n, (n_1, n_2, \ldots, n_k)) = \frac { n ! } { n _ { 1 } ! n _ { 2 } ! \cdots n _ { k } ! }$ to calculate permutation?

\item
\textbf{[combination]}
$\left( \begin{array} { l } { n } \\ { k } \end{array} \right) = \frac { n ! } { k ! ( n - k ) ! }$. Explain the equation on the left. What's the relationship between combination and permutation? Does the order matter in combination?

\item
\textbf{[probability properties]}
Use Venn Diagram to show $\operatorname { Pr } ( A \cup B ) = \operatorname { Pr } ( B ) + \operatorname { Pr } ( A ) - \operatorname { Pr } ( A \cap B )$. To generalize, what's $\operatorname { Pr } ( A \cup B \cup C)$?

\item
\textbf{[conditional probability]}
$\operatorname { Pr } ( B | A ) = \frac { \operatorname { Pr } ( A \cap B ) } { \operatorname { Pr } ( A ) }$. Explain the equation on the left. 
Can you make intuitive explanation for the joint probability $\operatorname { Pr } ( A \cap B ) = \operatorname { Pr } ( A ) \operatorname { Pr } ( B | A )$?

\item
\textbf{[law of total probability or marginalization]}
$\operatorname { Pr } ( B ) = \sum _ { i = 1 } ^ { n } \operatorname { Pr } \left( B \cap A _ { i } \right) = \sum _ { i = 1 } ^ { n } \operatorname { Pr } \left( A _ { i } \right) \operatorname { Pr } ( B | A _ { i } )$.

\item
\textbf{[independence]}
$\operatorname { Pr } ( A \cap B ) = \operatorname { Pr } ( A ) \operatorname { Pr } ( B )$. What condition must be satisfied for the equation on the left to hold?\\
Research on mutually independent and pair-wise independent.

\end{enumerate}
\end{document}