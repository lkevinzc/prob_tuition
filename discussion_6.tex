\documentclass[a4paper]{article}

\usepackage{parskip} % Package to tweak paragraph skipping
\usepackage{tikz} % Package for drawing
\usepackage{amsmath}
\usepackage{amssymb}
\usepackage{hyperref}
\usepackage{graphicx}
\usepackage{wrapfig}

\usepackage[a4paper,margin=0.2in]{geometry}
\usepackage{fancyhdr}
\pagestyle{fancy}


\title{EE2012/ST2334 Discussion 6}
\author{Liu Zichen}
\date{\today}
\begin{document}

\maketitle

\begin{enumerate}

\item
\textbf{[Estimation based on Normal Distribution]}
Given the observed data $x_{1}, x_{2}, \cdots, x_{n}$, we want to estimate the parameter $\theta$ which controls the distribution $f_{X}(x | \theta)$. A statistic is a function of the random variable which \textit{does not depend on any unknown parameters}. The statistic that one uses to obtain a point estimate is call an \textbf{estimator}. Interval estimation is to define two statistics and use their interval to estimate the parameters.

Unbiased estimator: $E(\widehat{\Theta})=\theta$.

Confidence interval for interval estimation\footnote{p21}. For the given error margin, the sample size is given by $n \geq\left(Z_{\alpha / 2} \frac{\sigma}{e}\right)^{2}$.

Confidence interval for the mean in 1) known variance case; 2) unknown variance case.

Confidence interval for the difference between two means. $\overline{X}_{1}-\overline{X}_{2}$ is a point estimator of $\mu_{1}-\mu_{2}$. Also two cases, known variances and unknown variances.

Confidence interval for the difference between two means for paired data (dependent data).

Confidence interval for a variance.

Confidence interval for the ratio of two variances with unknown means.

\item
\textbf{[Hypotheses testing based on Normal Distribution]}
\begin{itemize}
    \item Often, hypothesis is stated in a form that hopefully will be rejected, denoted as $H_0$ (Null hypothesis); its opposite (the one we need to accept due to insufficient data for concluding false) is denoted as $H_1$(Alternative hypothesis).
    
    \item Two tailed test and one tailed test.
    
    \item Type I and Type II error.
    
     \item Acceptance and rejection regions, critical value.
     
     \item Hypothesis testing on mean with known/unknown variance.
     \begin{itemize}
         \item Two-sided
         \item One-sided
     \end{itemize}
     
     \item Hypothesis testing on difference between two means.
     
     \item Hypothesis testing on variance.
     
     \item Hypothesis testing on ratio variance.
    
    
    
\end{itemize}













\end{enumerate}
\end{document}