\documentclass[a4paper]{article}

\usepackage{parskip} % Package to tweak paragraph skipping
\usepackage{tikz} % Package for drawing
\usepackage{amsmath}
\usepackage{amssymb}
\usepackage{hyperref}
\usepackage{graphicx}
\usepackage{wrapfig}

\usepackage[a4paper,margin=0.2in]{geometry}
\usepackage{fancyhdr}
\pagestyle{fancy}


\title{EE2012/ST2334 Discussion 3\footnote{Some examples from bookmarked pages of the lecture notes}}
\author{Liu Zichen}
\date{\today}
\begin{document}

\maketitle

\begin{enumerate}

\item
\textbf{[2-D random variable]}
Let $E$ be an experiment and $S$ a sample space associated with $E$. Let $X$ and $Y$ be two functions each assigning a real number to each $s \in S$. We call $(X, Y)$ a two-dimensional random variable (or random vector). Its \textbf{range} is $R_{X, Y}=\{(x, y) | x=X(s), y=Y(s), s \in S\}$. The definition can be extended to higher dimension, and they can be defined for both discrete and continuous random variable.

\item
\textbf{[Joint probability function]}

Discrete:
\begin{equation}
\begin{split}
    f_{X, Y}\left(x_{i}, y_{j}\right) & = 
    \operatorname{Pr}\left(X=x_{i}, Y=y_{j}\right) \\
    f_{X, Y}\left(x_{i}, y_{j}\right) & \geq 0 \text{ for all } (x_i, y_j) \in R_{X, Y}\\
    \sum_{i=1}^{\infty} \sum_{j=1}^{\infty} f_{X, Y}\left(x_{i}, y_{j}\right)& =\sum_{i=1}^{\infty} \sum_{j=1}^{\infty} \operatorname{Pr}\left(X=x_{i}, Y=y_{j}\right)=1
\end{split}
\end{equation}
Continuous: change $\sum$ to $\int$.

\item
\textbf{[Marginal distribution]}
Recall the law of total probability (discrete case):
\begin{equation}
    \begin{split}
        \operatorname{Pr}(A)
        &=\sum_{n} \operatorname{Pr}\left(A \cap B_{n}\right) \\
        &=\sum_{n} \operatorname{Pr}(A | B_{n}) \operatorname{Pr}\left(B_{n}\right)
    \end{split}
\end{equation}
If $(X, Y)$ is a 2-D discrete random variable, and its joint probability is $f_{X, Y}(x, y)$, the marginal distributions are:
\begin{equation}
    \begin{split}
        f_{X}(x)=\sum_{y} f_{X, Y}(x, y) \\
        f_{Y}(y)=\sum_{x} f_{X, Y}(x, y)
    \end{split}
\end{equation}

\item
\textbf{[Conditional distribution]}
\begin{equation}
    f_{Y|X}(y | x)=\frac{f_{X, Y}(x, y)}{f_{X}(x)}, \text { if } f_{X}(x)>0
\end{equation}

\item
\textbf{[Independence]}
\begin{equation}
    \mathrm{P}(A \cap B)=\mathrm{P}(A) \mathrm{P}(B) \Longleftrightarrow \mathrm{P}(A)=\frac{\mathrm{P}(A \cap B)}{\mathrm{P}(B)}=\mathrm{P}(A | B)
\end{equation}

\item
\textbf{[Expectation]}
\begin{equation}
    E[g(X, Y)]=\left\{\begin{array}{l}{\sum_{x} \sum_{y} g(x, y) f_{X, Y}(x, y), \quad \quad \quad \text { for Discrete } \mathrm{RV}^{\prime} \mathrm{s}} \\ {\int_{-\infty}^{\infty} \int_{-\infty}^{\infty} g(x, y) f_{X, Y}(x, y) d x d y, \text { for Continuous RV's }}\end{array}\right.
\end{equation}

\begin{enumerate}
    \item 
    A special case is that when $g(X, Y)=\left(X-\mu_{X}\right)\left(Y-\mu_{Y}\right)$, the expectation is the \textbf{covariance} of $(X, Y)$.
    $\operatorname{Cov}(X, Y)=E\left[\left(X-\mu_{X}\right)\left(Y-\mu_{Y}\right)\right] = E(X Y)-\mu_{X} \mu_{Y}$.
    
    \item
    If $X$ and $Y$ are independent, $\operatorname{cov}(X, Y)=0$. But $\operatorname{cov}(X, Y)=0$ does \textbf{NOT} imply independence.
    
    \item
    $\operatorname{Cov}(a X+b, c Y+d)=a c \operatorname{Cov}(X, Y)$, $V(a X+b Y)=a^{2} V(X)+b^{2} V(Y)+2 a b \operatorname{Cov}(X, Y)$
\end{enumerate}

\item
\textbf{[Correlation coefficient]}
\begin{equation}
    \rho_{X, Y}=\frac{\operatorname{Cov}(X, Y)}{\sqrt{V(X)} \sqrt{V(Y)}}
\end{equation}

\begin{enumerate}
    \item 
    $-1 \leq \rho_{X, Y} \leq 1$
    
    \item
    $\rho_{X, Y}$ measures the degree of \textbf{linear relationship} between $X$ and $Y$.
    
    \item
    If $X$ and $Y$ are independent, $rho_{X, Y} = 0$. But $rho_{X, Y} = 0$ does \textbf{NOT} imply independence.
\end{enumerate}

 
\end{enumerate}
\end{document}